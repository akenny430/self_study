% \documentclass[8pt, aspectratio=169]{beamer} 
\documentclass[8pt, aspectratio=149]{beamer} 

\usepackage{xcolor} % needed for defining colors

% \definecolor{mb}{HTML}{0071ce} % blue
\definecolor{mblue}{HTML}{2a3b87} % blue
% \definecolor{mgreen}{HTML}{009E73} % green
\definecolor{mgreen}{HTML}{7a7a44} % green
\definecolor{mred}{HTML}{ff5833} % red-orange

\definecolor{mra}{HTML}{ff6347}
\definecolor{mrb}{HTML}{b30000}
\definecolor{mrc}{HTML}{960018}
\definecolor{moa}{HTML}{ffa500}
\definecolor{mya}{HTML}{fada5e}
\definecolor{mga}{HTML}{9acd32}
\definecolor{mgb}{HTML}{4b8b3b}
\definecolor{mba}{HTML}{7ec0ee}
\definecolor{mbb}{HTML}{2a3b87}
\definecolor{mbc}{HTML}{0000ff}
%
\definecolor{dgray}{HTML}{aaaaaa}
\definecolor{mgray}{HTML}{c1c1c1}
\definecolor{lgray}{HTML}{dfdfdf}

\definecolor{bwhite}{HTML}{ffffff}
\definecolor{bgray}{HTML}{2c2c2c}

\definecolor{tblack}{HTML}{000000}
\definecolor{twhite}{HTML}{f2f2f2}

\definecolor{dred}{HTML}{960018}
\definecolor{dpink}{HTML}{ff9194}

% \definecolor{defcol}{HTML}{960018}
% \definecolor{defcol}{HTML}{afeeee}
% \definecolor{defcol}{HTML}{ff9194}

% setting color for beamer 
% this should be put into own file 
% \newcommand{\setTheColor}[5]
% % \newcommand{\setTheColor}[4]
% {
%     \colorlet{bgcol}{#1}
%     \colorlet{defcol}{#2}
%     \colorlet{textcol}{#3}
%     \colorlet{maincol}{#4}
%     \newcommand{\whichlogo}{../../../adminStuff/logos/#5}
% }   

%% different ways of emphasizing text 
\newcommand{\mDanger}[1]{\textcolor{mrc}{\textit{#1}}}
\newcommand{\mWarn}[1]{\textcolor{moa}{\textit{#1}}}
\newcommand{\mEmph}[1]{\textcolor{mgb}{\textit{#1}}}
\newcommand{\mDef}[1]{\textcolor{mba}{\textit{#1}}} 
%% latex commands need to be in camel case

%% custom commands to make my own vectors/ matrices, and tensors 
\newcommand{\mVector}[1]
{
    \ifcat\noexpand#1\relax
        \bm{#1}
    \else
        \mathbf{#1}
    \fi
}
\newcommand{\mMatrix}[1]{\mathbf{#1}} % matrix 
\newcommand{\mTensor}[1]{\bm{\mathscr{#1}}} % higher dimensional tensor 
\newcommand{\rVector}[1]{\bm{#1}} % matrix with random elements 

%% commands to have blackboard fonts for standard probability stuff (e.g. expected value, variance)
\newcommand{\pr}{\mathbb{P}}
\newcommand{\ev}{\mathbb{E}}
\newcommand{\var}{\mathbb{V}\mathrm{ar}}
\newcommand{\sd}{\mathrm{sd}}
\newcommand{\cov}{\mathbb{C}\mathrm{ov}}
\newcommand{\cor}{\mathbb{C}\mathrm{or}}
\newcommand{\bias}{\mathbb{B}\mathrm{ias}}

%% common sets used 
\newcommand{\N}{\mathbb{N}} %% real numbers 
\newcommand{\Z}{\mathbb{Z}} %% real numbers 
\newcommand{\R}{\mathbb{R}} %% real numbers 
\newcommand{\C}{\mathbb{C}} %% complex numbers 

%% common operators 
\newcommand{\diff}{\mathrm{d}} %% differential 
\newcommand{\id}{\,\diff} %% differential (used in integral)  
% \newcommand{\T}{T} %% transpose 
\newcommand{\T}{\top} %% transpose 

%% renewing binomial coefficient, this looks better 
\renewcommand{\binom}[2]{
    \begin{pmatrix}
        #1 \\ #2
    \end{pmatrix}
}

%% command for indicator functions, need bb font for numbers 
\usepackage{bbm}
\newcommand{\ind}{ \mathbbm{1} } % symbol for indicator function 

%% defining arg-min and arg-max functions
\DeclareMathOperator*{\argmax}{argmax}
\DeclareMathOperator*{\argmin}{argmin}

%% command for iid sample 
\newcommand{\iid}{\overset{\mathrm{iid}}{\sim}} 

%% general subsetting for vector/matrices 
\newcommand{\elSub}[3]{{#1}_{#2,#3}} 

%' ============================================================================================================
%  FOOTER
%' ============================================================================================================

% \def\mysep{\(\diamond\)}
\def\mysep{\(\circ\)}



\makeatletter
\setbeamertemplate{footline}
{
    \dimen0 = \paperwidth
    \multiply\dimen0 by \insertframenumber
    \divide\dimen0 by \inserttotalframenumber
    \edef\progressbarwidth{\the\dimen0}
    
    {\leavevmode%
    \hbox{%
    \begin{beamercolorbox}[wd = 0.805\paperwidth, ht = 4ex , dp = 2ex, left]{section in head/foot}%
        \ifx\insertauthor\empty~~~~ %% if no author, leave blank, else put all info 
        \else~~~~\faCopyright~~\insertauthor{}~~~~\href{https://github.com/akenny430}{\faGithub~a{}ken{}n{}y{}430}~~~~\href{https://www.linkedin.com/in/akenny430/}{\faLinkedin~a{}ken{}n{}y{}430}
        \fi
    \end{beamercolorbox}%
    %% numbers in the slides (choose which one) 
    \begin{beamercolorbox}[wd = 0.195\paperwidth, ht = 4ex, dp = 2ex, right]{section in head/foot}%
        % \hspace{0.25in}\insertframenumber ~/ \inserttotalframenumber~~~~  %% frame / total 
        \hspace{0.25in}\insertframenumber~~~~~ %% frame only 
    \end{beamercolorbox}%
    }%
    \vskip0pt%
    }%
    %% color bar on the bottom (if using only full bar, need to change color opacity later on) 
    \begin{beamercolorbox}[wd = \paperwidth, ht = 0.6ex, dp = 0ex]{author in head/foot}%
        % \begin{beamercolorbox}[wd = \progressbarwidth, ht = 0.6ex, dp = 0ex]{title in head/foot}%
        % \end{beamercolorbox}%
    \end{beamercolorbox}%
}
\makeatother



%' ============================================================================================================
%  TITLE
%' ============================================================================================================

\usepackage{fontawesome}
\defbeamertemplate*{title page}{customized}[1][]
{
    \centering
    \vspace{0.5in}
    \ifx\insertsubtitle\empty {\LARGE\inserttitle} \par \vspace{2em}
    \else {\LARGE\inserttitle} \par \vspace{0.1in} {\insertsubtitle} \par \vspace{1em}
    \fi
    
    % {\insertauthor}    \par\vspace{0.5em}
    % {\insertinstitute} \par\vspace{0.4em}
    % {\insertdate}      \par%\vspace{0.3em}
    

    % \vspace{4em}
    % ~~~\href{https://github.com/akenny430}{\faGithub~ a{}ken{}n{}y{}430}
    % ~~~\href{https://www.linkedin.com/in/akenny430/}{\faLinkedin~ a{}ken{}n{}y{}430}
    % ~~~\href{https://www.reddit.com/user/akenny430} {\faRedditAlien~ u/a{}ken{}n{}y{}430}
}





%' ============================================================================================================
%  SECTIONS
%' ============================================================================================================

\AtBeginSection[]{
    \begin{frame}{}
    
    \centering
    \vfill
    \vspace{0.5in}
    \LARGE\insertsection
    \vfill
    
    \end{frame}
}




%' ============================================================================================================
%  GENERAL
%' ============================================================================================================

%% want way to override shit spacing with sub itemize stuff 
\setbeamersize{text margin left = 0.25in, text margin right = 0.25in} %% indents 

\setbeamertemplate{navigation symbols}{} %% getting rid of nav symbols on bottom 

%% itemize items 
\setbeamertemplate{itemize item}[circle]{}
\setbeamercolor{itemize item}{fg = textcol}

%% itemize subitems 
\setbeamertemplate{itemize subitem}[square]{}
\setbeamercolor{itemize subitem}{fg = textcol}

%% enumerate items 
\setbeamertemplate{enumerate item}[default]{}
\setbeamercolor{enumerate item}{fg = textcol}

% \hypersetup{colorlinks = false}




%' ============================================================================================================
%  COLORS
%' ============================================================================================================

%% command for setting colors of beamer slides (main, text, background, definition) 
% \newcommand{\setTheColor}[5]
\newcommand{\setTheColor}[4]
{
    \colorlet{bgcol}{#1}
    \colorlet{defcol}{#2}
    \colorlet{textcol}{#3}
    \colorlet{maincol}{#4}
    % \newcommand{\whichlogo}{path/to/image/#5} %% comment this out if there is a logo to put in the top right 
}   

%% slide title 
\setbeamercolor{frametitle}{bg = bgcol, fg = textcol}   
% \setbeamercolor{frametitle}{bg = maincol!30, fg = textcol}      
% \setbeamercolor{frametitle}{bg = bgcol, fg = maincol}           

%% increasing progress bar 
\setbeamercolor{title in head/foot}{bg = maincol, fg = textcol} 

%% background progress bar 
% \setbeamercolor{author in head/foot}{bg = maincol!50, fg = textcol} 
\setbeamercolor{author in head/foot}{bg = maincol, fg = textcol} 

%% text in footer 
\setbeamercolor{section in head/foot}{bg = bgcol, fg = textcol} 

%% background color 
\setbeamercolor{background canvas}{bg = bgcol} 

%% text color 
\setbeamercolor{normal text}{fg = textcol} 






%' ============================================================================================================
%' LOGO
%' ============================================================================================================

%' un-comment this whole file to include the logo 
% \usepackage{eso-pic}
% \newcommand\AtPagemyUpperLeft[1]{\AtPageLowerLeft{%
% \put(\LenToUnit{0.9\paperwidth},\LenToUnit{0.9\paperheight}){#1}}}
% \AddToShipoutPictureFG{
%   \AtPagemyUpperLeft{{\includegraphics[width=4em,keepaspectratio]{\whichlogo}}}
% }%





%' ============================================================================================================
%' ============================================================================================================

% packages used to adjust formatting of document
\usepackage{microtype} %% don't remember why this is important, but makes it look nicer? 

%% choose which style for math font 
% \usepackage[light]{roboto} %% roboto font 
\usefonttheme[onlymath]{serif} %% anything not math is serif 
% \usefonttheme[stillsansserifsmall]{serif} %% only text on bottom is serif 

%% making tables
\usepackage{booktabs}
\usepackage{multirow, multicol}
\usepackage{tabularx}

%% math
\usepackage{mathtools, amssymb, bm} 
\usepackage[mathscr]{euscript}

%% images
\usepackage{graphicx}
\usepackage{xcolor}

%% extra
\usepackage{tikzsymbols}
    
%% changing emph definition 
%% should make own proper definition commands (look into theorem environment)
\renewcommand{\emph}[1]{\textcolor{defcol}{\textsl{#1}}}

%% color boxes for theorem environment 
\usepackage{tcolorbox}
% \tcbset{colback=mblue!5, colframe=mblue!75!black, boxrule=0.25mm}
\tcbset{colback=lgray!50, colframe=mblue!75!black, boxrule=0.25mm}


%' ============================================================================================================
%  OTHER BEAMER STUFF 
%' ============================================================================================================

% vertical spacing 
% \newcommand{\vs1}{\vspace{1em}}
% \newcommand{\vs2}{\vspace{2em}}
 
\setTheColor{bwhite}{dred}{tblack}{mblue} %% (white background)  
% \setTheColor{bgray}{dred}{twhite}{mblue} %% (gray background)

\begin{document}
    
\title{Introduction to differentials} 
\author{Aiden Kenny} 
\institute{Nothing} 
\date{2022}

% \newcommand{\vs1}{\vspace{1em}}
% \newcommand{\vs2}{\vspace{2em}}
\newcommand{\thel}{t{}h} 
\newcommand{\Jacobian}{J{}a{}c{}o{}b{}i{}a{}n}

%' ============================================================================================================================================================
\begin{frame}{}

    \maketitle

\end{frame}

%' ============================================================================================================================================================
\begin{frame}{Outline}

    \tableofcontents

\end{frame}



\section{Introduction} 
%' ============================================================================================================================================================
\begin{frame}{Derivatives} 

    The \mDef{derivative} of a function \( f(x) : \R \to \R \) is defined to be 
    \begin{align*}
        f'(x)
        \coloneqq 
        \lim_{\delta \to 0} \frac{f(x + \delta) - f(x)}{\delta}. 
    \end{align*}
    \begin{itemize}
        \item ``If you make a very small change in \( x \), how will \( f \) change?'' 
        \item And this is then made rigorous as the small change in \( x \) becomes small from the limit. 
    \end{itemize}

\end{frame}



%' ============================================================================================================================================================
\begin{frame}{Leibniz notation}

    Intro calculus classes teach of Leibniz notation: \( \diff f / \diff y \) as another way to write \( f'(x) \): 
    \begin{align*}
        \frac{\diff f}{\diff x} = f'(x) 
        = \lim_{\delta \to 0} \frac{f(x + \delta) - f(x)}{\delta}
    \end{align*}
    \begin{itemize}
        \item Intuition: \( \diff x \) is an ``infinitely small'' change in \( x \), \( \diff f \) is an ``infinitely small'' change in \( f \) 
        \item Then taking the ratio gives the ``instantaneous'' rate of change of \( f \): the derivative     
        % \item And then learn \mDanger{``this is not a fraction!''} 
        % \item An ``infinitely small'' change is not rigorously defined 
    \end{itemize}

    \vspace{1em}
    \mWarn{We will ignore this way of thinking} 
    \begin{itemize}
        \item Instead, give \textit{differentials} their own rigorous definition in general  
        \item Then make connection to derivatives (if function is differentiable) 
    \end{itemize} 

\end{frame}



%' ============================================================================================================================================================
\begin{frame}{Partial derivatives}

    For a function \( \mVector{f}(\mVector{x}) : \R^m \to \R^n \) (now multivariate), 
    the \mDef{partial derivative} is  
    \begin{align*}
        \frac{\partial f_i}{\partial x_j} 
        \coloneqq 
        \lim_{\delta \to 0} \frac{f_i(\mVector{x} + \delta \mVector{e}_j) - f_i(\mVector{x})}{\delta},
    \end{align*} 
    where \( \mVector{e}_j \in \R^m \) is the unit vector where the \( j \)\thel{} element is \( 1 \) and the rest are \( 0 \). 
    \begin{itemize}
        \item Looking at how \( i \)\thel{} component of \( \mVector{f} \) changes with respect to \( j \)\thel{} component of \( \mVector{x} \).     
    \end{itemize}  
    
    \vspace{1em} 
    Think of being acted on by the \textit{differential operator} \( \partial / \partial x_j \): 
    \begin{align*}
        \frac{\partial f_i}{\partial x_j} = \frac{\partial}{\partial x_j} f_i(\mVector{x}),
    \end{align*} 
    we are ``acting'' on \( f_i(\mVector{x}) \) by differentiating with respect to \( x_j \) 
    \begin{itemize}
        \item \mDanger{This operator is not a fraction! It is just the notation chosen to represent this action} 
    \end{itemize}

\end{frame}



%' ============================================================================================================================================================
\begin{frame}{...}

    If \( f(x) : \R \to \R \), then the partial derivative is 
    \begin{align*}
        \frac{\partial f}{\partial x} 
        = \lim_{\delta \to 0} \frac{f(x + \delta \cdot 1) - f(x)}{\delta} 
        = f'(x).  
    \end{align*} 
    The partial derivative is just... the derivative for univariate functions. 

    \vspace{1em} 
    \mEmph{We will use the partial derivative notation when differentiating univarite functions} 
    \begin{itemize}
        \item Better for generalizing 
    \end{itemize}

\end{frame}



%' ============================================================================================================================================================
\begin{frame}{Jacobian matrix}

    For a function \( \mVector{f}(\mVector{x}) : \R^m \to \R^n \)
    put all partial derivatives into a matrix 
    \begin{align*}
        \frac{\partial \mVector{f}}{\partial \mVector{x}} 
        % = \mMatrix{J}_{\mVector{f}}(\mVector{x}) 
        \coloneqq 
        \begin{bmatrix}
            \partial f_1 / \partial x_1 & \cdots & \partial f_1 / \partial x_m \\ 
            \vdots & \ddots & \vdots \\ 
            \partial f_n / \partial x_1 & \cdots & \partial f_n / \partial x_m \\ 
        \end{bmatrix} 
        \in \R^{n \times m}. 
    \end{align*}
    This is commonly called: 
    \begin{itemize}
        \item The \textit{derivative} (understanding it is a combination of all partial derivatives). 
        \item The \textit{\Jacobian} of \( \mVector{f} \) with respect to \( \mVector{x} \).   
        \item Sometimes labeled as \( \mMatrix{J}_{\mVector{x}}\mVector{f} \) or \( \mVector{J} \) 
    \end{itemize}

    \vspace{1em}
    \mEmph{When we say ``the derivative'' of \( \mVector{f}(\mVector{x}) \), we are talking about the collection of all partial derivatives} 

\end{frame}



%' ============================================================================================================================================================
\begin{frame}{Special cases of Jacobian}

    If \( f(x) : \R \to \R \), then \( \partial f / \partial x \in \R^{1 \times 1} = \R \), so it is just ``the derivative'' 
    
    \vspace{2em} 
    If \( f(\mVector{x}) : \R^m \to \R \), then \( \partial f / \partial \mVector{x} \in \R^{1 \times m} \) is a \textit{row vector}  
    \begin{itemize}
        \item Aside: this is not the same as the \textit{gradient}, \( \nabla_{\mVector{x}}f \), which is a column vector 
        \item So \( (\partial f / \partial \mVector{x})^\T = \nabla_{\mVector{x}}f \), they are transposes of each other 
        % \item This is why the gradient is used more 
    \end{itemize}

    \vspace{2em} 
    If \( \mVector{f}(x) : \R \to \R^n \), then \( \partial \mVector{f} / \partial x \in \R^{n \times 1} = \R^n \) is a \textit{column vector}  

\end{frame}



%' ============================================================================================================================================================
\begin{frame}{Taylor series}

    For a function \( f : \R \to \R \in \mathcal{C}^d \) (can be differentiated \( d \) times), 
    % where \( d = \N \cup \{\infty\} \), 
    it's \mDef{Taylor series centered at \( x_* \)} is  
    \begin{align*}
        f(x) 
        % = \sum_{i = 0}^d \frac{f^{(i)}(x_*)}{i!} (x - x_*)^i 
        % = \sum_{i = 0}^d \frac{1}{i!} \frac{\partial^i f}{\partial x^i}(x_*) (x - x_*)^i
        = \sum_{i = 0}^d \frac{1}{i!} \frac{\partial^i f(x_*)}{\partial x^i} (x - x_*)^i 
        = f(x_*) + \frac{\partial f(x_*)}{\partial x} (x - x_*) + \cdots 
    \end{align*}  
    % (we wrote out both notations of the derivative)

    \vspace{1em} 
    For function \( \mVector{f}(\mVector{x}) : \R^m \to \R^n \), Taylor series centered at \( \mVector{x}_* \) is 
    \begin{align*}
        \mVector{f}(\mVector{x}) 
        = \mVector{f}(\mVector{x}_*) + \frac{\partial \mVector{f}(\mVector{x_*})}{\partial \mVector{x}} ( \mVector{x} - \mVector{x}_* ) + \cdots 
    \end{align*} 

    \begin{itemize}
        \item \mEmph{We are mainly concered with the linear component of the Taylor series} 
        \item Other terms of multivariate Taylor series are outside of scope 
    \end{itemize}

\end{frame}



\section{Differentials} 
%' ============================================================================================================================================================
\begin{frame}{Differentials}

    A \mDef{differential} for any function \( f(x) : \R \to \R \) is defined as 
    \begin{align*}
        \diff f 
        = \diff f(x; \Delta) 
        % \coloneqq \lim_{\delta \to 0} \Big( f(x + \delta) - f(x) \Big). 
        \coloneqq f(x + \Delta) - f(x). 
    \end{align*} 
    Informally, is just the difference when you have a very small change in the input 
    \begin{itemize}
        \item \( \Delta \) is usually taken to be very small (\( \Delta < 0.01 \)), \mWarn{but \( \Delta \neq 0 \)!} 
        % \item \mEmph{\( \Delta \) should be very small, and }
        \item So the differential will depend on the value of \( \Delta \) 
        \item It is a function itself, \( \diff f : \R \to \R \)  
    \end{itemize}

    \vspace{1em}
    For special case \( f(x) = x \), 
    \begin{align*}
        \diff f = (x + \Delta) - x = \Delta \coloneqq \diff x. 
    \end{align*}
    The output is just the input, make this the differential of the input. 
    \begin{itemize}
        \item Will write \( \Delta \) as \( \diff x \) doing forward 
    \end{itemize}

\end{frame}


%' ============================================================================================================================================================
\begin{frame}{Multivariate differentials}

    If \( \mVector{f}(\mVector{x}): \R^m \to \R^n \), a multivariate differential is just a vector: 
    \begin{align*}
        \mVector{\Delta} = \diff \mVector{x} 
        = \begin{bmatrix}
            \diff x_1 \\ \vdots \\ \diff x_m
        \end{bmatrix}, 
        ~~~
        \diff \mVector{f} = \mVector{f}(\mVector{x} + \diff \mVector{x}) - \mVector{f}(\mVector{x}) 
        = \begin{bmatrix}
            \diff f_1 \\ \vdots \\ \diff f_n 
        \end{bmatrix}
    \end{align*}

\end{frame}



\section{Connection to derivatives} 
%' ============================================================================================================================================================
\begin{frame}{Relating \( \diff f \) to \( \diff x \)}

    For a function \( f: \R \to \R \), we can have differentials for the input (\( \diff x \)) and output (\( \diff f \))  
    \begin{itemize}
        \item \mEmph{We would like to have a way to relate \( \diff f \) to \( \diff x \)} 
        \item ``If we change input by \( \Delta \), how will output change?'' 
        \item This sounds a lot like a derivative... \mWarn{but it does not need to be (e.g. Brownian motion)} 
    \end{itemize}

\end{frame}



%' ============================================================================================================================================================
\begin{frame}{Univariate differentiable functions}

    % If \( f: \R \to \R \) is differentiable at least once, the Taylor series of \( f(x + \Delta) \) centered at \( x \) is 
    % \begin{align*}
    %     f(x + \Delta) 
    %     = \sum_{i = 1}^{\infty} \frac{1}{i!} \frac{\partial^i f(x)}{\partial x} (x + \Delta - x)^i 
    %     = f(x) + \frac{\partial f}{\partial x} \Delta + \frac{\partial^2 f}{\partial x^2} \frac{\Delta^2}{2} + \cdots.
    % \end{align*}
    % Differential is then 
    % \begin{align*}
    %     \diff f 
    %     = f(x + \Delta) - f(x) 
    %     = \frac{\partial f}{\partial x} \Delta + \frac{\partial^2 f}{\partial x^2} \frac{\Delta^2}{2} + \cdots
    %     = \frac{\partial f}{\partial x} \Delta + o(\Delta). 
    % \end{align*}
    % As \( \Delta \to 0 \), \mEmph{the other terms all shrink much faster, so we ignore them}: 
    If \( f: \R \to \R \) is differentiable at least once, the Taylor series of \( f(x + \diff x) \) centered at \( x \) is 
    \begin{align*}
        f(x + \diff x) 
        = \sum_{i = 1}^{\infty} \frac{1}{i!} \frac{\partial^i f(x)}{\partial x} (x + \diff x - x)^i 
        = f(x) + \frac{\partial f}{\partial x} \diff x + \frac{1}{2} \frac{\partial^2 f}{\partial x^2} \diff x^2 + \cdots.
    \end{align*}
    Differential is then 
    \begin{align*}
        \diff f 
        = f(x + \diff x) - f(x) 
        = \frac{\partial f}{\partial x} \diff x + \frac{\partial^2 f}{\partial x^2} \frac{\diff x^2}{2} + \cdots
        = \frac{\partial f}{\partial x} \diff x + o(\diff x). 
    \end{align*}
    As \( \diff x \to 0 \), \mEmph{the other terms all shrink much faster, so we ignore them}: 
    \begin{align*}
        \diff f 
        = \frac{\partial f}{\partial x} \diff x
    \end{align*}
    \mEmph{This is the connection of differentials for a differentiable function} 
    \begin{itemize}
        \item Only looking at linear part of Taylor series 
        \item Higher order terms technically still there, but are so small we can safely ignore 
        \item \mDanger{This will not always be the case (e.g. geometric brownian motion)} 
    \end{itemize}

\end{frame}



%' ============================================================================================================================================================
\begin{frame}{Multivariate differentiable functions}

    Same idea with multivariate function \( \mVector{f}(\mVector{x}): \R^m \to \R^n \): Taylor series of \( \mVector{f}(\mVector{x} + \diff \mVector{x}) \) 
    centered at \( \mVector{x} \) is 
    \begin{align*}
        \mVector{f}(\mVector{x} + \diff \mVector{x}) 
        = \mVector{f}(\mVector{x}) + \frac{\partial \mVector{f}}{\partial \mVector{x}} \diff \mVector{x} + \mVector{o}(\diff \mVector{x}). 
    \end{align*}
    As \( \diff \mVector{x} \to \mVector{0} \), the \( \mVector{o}(\diff \mVector{x}) \) can be ignored, 
    resulting in 
    \begin{align*}
        \diff \mVector{f} 
        = \frac{\partial \mVector{f}}{\partial \mVector{x}} \diff \mVector{x} 
    \end{align*}

    \vspace{1em} 
    This works for any dimensionality: 
    \begin{itemize}
        \item \( f(x): \R \to \R \) we already did 
        \item \( f(\mVector{x}): \R^m \to \R \) gives 
        \begin{align*}
            \diff f 
            = \frac{\partial f}{\partial \mVector{x}} \diff \mVector{x} 
            = \begin{bmatrix}
                \displaystyle \frac{\partial f}{\partial x_1} & \cdots & \displaystyle \frac{\partial f}{\partial x_m} 
            \end{bmatrix} \begin{bmatrix}
                \diff x_1 \\ \vdots \\ \diff x_m
            \end{bmatrix} 
            = \sum_{j = 1}^m \frac{\partial f}{\partial x_j} \diff x_j. 
        \end{align*}
        \item \( \mVector{f}(x): \R \to \R^n \), same idea 
    \end{itemize}

\end{frame}



%' ============================================================================================================================================================
\begin{frame}{}

    

\end{frame}



\end{document}
